\section{Introduction}
\label{sec:intro}
Tor is a distributed overlay network designed to anonymize TCP-based applications such as web browsing, secure shell, and instant messaging~\cite{dingledine2004tor}. Its goal is to protect the privacy of its users and defend them against traffic analysis, a form of network surveillance. In order to provide this anonymity, data packets are sent through a multi-hop virtual circuit of nodes called relays or onion routers. By using a separate set of encryption keys for each hop, these encrypted relay circuits ensure that no observer at a single point along the communication channel can directly identify both the true source and the final destination. While many users are using Tor for its intended purpose--to prevent Internet Service Providers (ISPs) from learning their online browsing habits, to aid law enforcement in sting operations, and to allow human rights activists to anonymously report abuses from danger zones~\cite{torusers}--it is no surprise that there are others using it to facilitate their illegal activities. This has made Tor a target for deanonymization attacks by governmental adversaries, such as the NSA and FBI, and ISPs whose aim is to deanonymize Tor users and link them back to their online activity. The Tor project openly welcomes academic research that helps defend it from such attacks as well as research that helps expose unknown vulnerabilities in its anonymous communication system.

Tor relies on the Transport Layer Security (TLS) protocol and asymmetric key cryptography to encrypt messages that travel between the nodes of the Tor relay circuits. In RSA public-key cryptography, the strength of the system depends on the degree of difficulty for which a generated private key can be determined from its corresponding public key. Factors such as the length of the key and the strength of the random number generator during key generation have proven to be critical to the security of modern cryptographic systems~\cite{rivest1978method, heninger2012mining, lenstra2012ron}. In 2012,~\cite{heninger2012mining, lenstra2012ron} conducted what were at the time the largest network surveys of TLS and SSH servers.~\cite{heninger2012mining} found an alarmingly large number of vulnerable RSA public keys and claimed to have computed the RSA private keys for 0.50\% of TLS hosts. The research group pinpointed the underlying issue to malfunctioning random number generators. More specifically,~\cite{heninger2012mining} traced the vulnerable keys to resource-constrained network devices which had not been properly seeded with adequate amounts of entropy. From their analysis of over 12 million TLS hosts and over 10 million SSH hosts,~\cite{heninger2012mining} identified three types of vulnerabilities: repeated keys, factorable RSA keys, and repeated ephemeral keys in DSA signatures. In all three cases, the problems were tied to specific faulty key generation implementations caused by insufficient entropy.

Tor has grown to become the most popular and well-researched anonymity network consisting of over 2 million directly connecting users and over 7,000 volunteer-operated Tor relay severs~\cite{metrics}. While many of the relays run on large PCs and VPS data centers, there is anecdotal evidence that some relays run on smaller devices such as Raspberry Pi products. This poses a potential risk to the anonymity of Tor because resource-constrained machines are more susceptible to generating weak RSA public keys~\cite{heninger2012mining, lenstra2012ron}. In order to detect potentially weak factorable keys in Tor, I used~\cite{heninger2012mining}'s fastgcd solution for efficiently computing the greatest common divisor for every pair of integers in a large set. While it is not computationally practical to factor large integer RSA keys, it is computationally practical to compute their pair-wise GCDs. Thus, if an attacker is able to find a pair of RSA key moduli with shared prime factors, then she can easily use this information to compute the private keys~\cite{heninger2012mining, lenstra2012ron}. Knowing the private key of a Tor relay server has serious security implications, and is a critical first step to compromising the anonymity of the system. With private key in hand, that adversary now has control of the relay.

In Section 2 I provide background. In Section 3 I discuss my methodology. In Section 4 I provide evaluation. In Section 5 I discuss future work, and in Section 6 I conclude.