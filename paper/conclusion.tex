\section{Conclusion}
\label{sec:conclusion}

Inspired by recent research that studied weak keys in deployed systems, we set
out to investigate if the Tor network suffers from similar issues.  To that
end, we drew on a public archive containing cryptographic keys of Tor relays
dating back to 2005, which we subsequently analyzed for weak RSA keys.  We
discovered that \first ten relays shared an RSA modulus, \second 3,557 relays
shared prime factors, and \third 122 relays used non-standard RSA exponents.

Having uncovered these anomalies, we then proceeded to characterize the
affected relays, tracing back the issues to mostly harmless experiments run by
academic researchers and hobbyists, but also to attackers that targeted Tor's
distributed hash table which powers onion services.  To learn more, we
implemented a tool that can determine what onion services were attacked by a
given set of malicious Tor relays, revealing four onion services that fell prey
to these attacks.

The practical value of our work is twofold.  First, our uncovering and
characterizing of Tor relays with anomalous keys provides an anatomy of
real-world attacks that The Tor Project can draw upon to improve its monitoring
infrastructure for malicious Tor relays.  Second, our work provides The Tor
Project with tools to verify the RSA keys of freshly set up relays, making the
network safer for its users.  In addition, onion service operators can use our
code to monitor their services and get notified if Tor relays make an effort to
deanonymize their onion service.  We believe that this is particularly useful
for sensitive deployments such as SecureDrop instances.
