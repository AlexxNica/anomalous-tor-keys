\section{Conclusion}
\label{sec:conclusion}

Previous research has studied the problem of weak RSA keys in different
systems, and we wondered if there might be weak keys in the Tor
network, too, that have the potential to compromise Tor users' safety.
Thus, the goal of our work was to look for weak and anomalous keys in Tor
and investigate their origins in order to address the problem.  We
achieved our goal by gathering all the archived RSA keys used in Tor since
2005 and examining them for common prime factors as previous work had done.
Additionally, we looked for keys that shared the same moduli and
for keys that had non-standard public exponents.

We found indications that entities had purposely created weak and anomalous keys
in order to attack Tor's onion services.  We also found that researchers
inadvertently created weak keys while conducting experiments on Tor.
Our work demonstrates that the presence of weak and anomalous RSA keys in Tor is
often a sign of malicious activity that should be paid attention to, and 
indeed, our findings motivated The Tor Project to develop scripts to look for
non-standard RSA exponents, which go against Tor's specification.
