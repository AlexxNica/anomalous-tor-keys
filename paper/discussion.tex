\section{Discussion}
\label{sec:discussion}
\textbf{Implications of weak Tor keys}
The implications of factorable RSA indentity keys in Tor relay servers are serious. As discussed earlier, discovering that a relay's identity key is factorable means that an attacker can compromise the key pair by gaining access to the private key. To fully prove the brokenness of the vulnerable public keys we found, we calculated all 3,557 corresponding private keys. We then used these private keys to forge evil signatures, allowing us to impersonate the relays.

As touched on earlier in Section~\ref{sec:tor-network}, the main use of the identity key in Tor is to sign the relay's ``descriptor.'' This document includes various information about the relay e.g. its IP address, contact information, etc. Since the identity key of every relay is included in the Tor consensus, all clients know the identity keys of all relays. The consensus acts as the public key infrastructure (PKI) of Tor. When a client fetches a Tor descriptor, it checks the consensus for the right identity key, and then verifies the descriptor signature. This ensures that the descriptor was created by an entity that knew the private identity key, and hence is not a fake descriptor created by a third party. If an attacker were to break the identity key of a relay (as we demonstrated), she could start signing descriptors in its name and publishing them. The adversary could publish whatever information she wanted in the descriptor e.g. her own IP address, keys, etc. Tor clients would be fooled into thinking it's the actual relay, but instead, would be connecting to the attacker. Thus, one can imagine an exploit that could hack most identify keys of Tor relays, allowing the attacker to redirect all clients to her own machines and monitor their traffic.

Factoring the identity key is equivalent to controlling the compromised Tor relay. While it does not immediately allow an adversary to link a Tor user's IP address to her destination, it is a critical step that can bring the attacker much closer to that goal. We have developed a set of tools that can help Tor developers detect such weak keys. The tools are publically available at \url{https://github.com/citp/weak-tor-keys/tree/master/code}.

\textbf{Tor keys with non-standard exponents} Recall that the Tor reference implementation hard-codes its public RSA exponent
to 65,537~\cite[\S~0.3]{torspec}.  The Tor Project could prevent non-standard
exponents by having the directory authorities reject relays whose descriptors
have an RSA exponent other than 65,537, thus slowing down the search for
fingerprint prefixes.  Adversaries would then have to iterate over the primes
$p$ or $q$ instead of the exponent, making the search more expensive.  Given
that we discovered only 122 unusual exponents in over ten years, we believe that
rejecting non-standard exponents is a viable defense in depth.

Future work should shed light on the public keys of onion services.  Onion
services have an incentive to modify their fingerprints to make them both
recognizable and easier to remember.  Facebook, for example, was lucky enough to
obtain the easy-to-remember name \url{facebookcorewwwi.onion}~\cite{facebook}.
The tool Scallion assists onion service operators in creating such vanity
domains~\cite{scallion}.  It remains an open question whether vanity domains
benefit usability, harm security, or a combination
thereof~\cite{vanity-domains}.  Unlike the public keys of relays, onion service
keys are not archived, so a study would have to begin with actively fetching
onion service keys.

\textbf{\textit{in vivo} Tor Research} For the purposes of completeness, we feel compelled to mention \textcolor{red}{wording}. The continued robustness of Tor against potential attackers can be partly attributed to the small army of academic researchers who have \textcolor{red}{wording}. For this reason, the Tor project welcomes academic research that helps expose unknown vulnerabilities in its anonymity system \cite{torresearch}. However, caution must be taken when conducting research \textit{in vivo}, that is, when conducting research using the live Tor network. In Section~\ref{sec:shared-primes}, we showed how a small mistake in key generation led to many weak keys in the Tor network. In order to help keep researchers and users safe, The Tor Project has recently launched a research safety board whose aim is to assist researchers in safely conducting Tor measurement studies \cite{vanity-domains}, e.g. by running experiments in private Tor networks that are controlled by the researchers, or by using network simulators such as Shadow \cite{stem}.
