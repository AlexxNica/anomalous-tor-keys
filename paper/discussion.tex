\section{Discussion}
\label{sec:discussion}
\subsection{Implications of anomalous Tor keys}
As touched on earlier in Section~\ref{sec:tor-network}, the main use of the
identity key in Tor is to sign the relay's descriptor, which includes various
information about the relay, \eg, its IP address, contact information, \etc
Relays publish their public identity keys in their descriptor.  The network
consensus acts as the public key infrastructure of Tor.  Signed by the directory
authorities whose public keys are hard-coded in Tor's source code, the network
consensus points to the descriptors of each Tor relay that is currently online.
If an attacker were to break the identity key of a relay (as we demonstrated),
she could start signing descriptors in the relay's name and publishing them. The
adversary could publish whatever information she wanted in the descriptor, \eg
her own IP address, keys, \etc, in order to fool Tor clients.

\subsection{Preventing non-standard exponents}
Recall that the Tor reference implementation hard-codes its public RSA exponent
to 65,537~\cite[\S~0.3]{torspec}.  The Tor Project could prevent non-standard
exponents by having the directory authorities reject relays whose descriptors
have an RSA exponent other than 65,537, thus slowing down the search for
fingerprint prefixes.  Adversaries would then have to iterate over the primes
$p$ or $q$ instead of the exponent, rendering the search computationally
more expensive because of the cost of primality tests.  Given that we discovered
only 122 unusual exponents in over ten years of data, we believe that rejecting
non-standard exponents is a viable defense in depth.

\subsection{Analyzing onion service public keys}
Future work should shed light on the public keys of onion services.  Onion
services have an incentive to modify their fingerprints to make them both
recognizable and easier to remember.  Facebook, for example, was lucky to
obtain the easy-to-remember onion domain
\url{facebookcorewwwi.onion}~\cite{facebook}.  The tool Scallion assists onion
service operators in creating such vanity domains~\cite{scallion}.  The
implications of vanity domains on usability and security are still poorly
understood~\cite{vanity-domains}.  Unlike the public keys of relays, onion
service keys are not archived, so a study would have to begin with actively
fetching onion service keys.

\subsection{\textit{In vivo} Tor research}
Caution must be taken when conducting research using the live Tor network.
Section~\ref{sec:shared-primes} showed how a small mistake in key generation led
to many vulnerable Tor relays.  To keep its users safe, The Tor Project has
recently launched a research safety board whose aim is to assist researchers in
safely conducting Tor measurement studies~\cite{safety-board}.  This may entail
running experiments in private Tor networks that are controlled by the
researchers, or using network simulators such as Shadow~\cite{Jansen2012a}.

\subsection{The effect of next-generation onion services}
As of August 2017, The Tor Project is finalizing the design of next-generation
onion services~\cite{prop224}.  In addition to stronger cryptographic
primitives, the design fixes the issue of predicting an onion service's location
in the hash ring by incorporating a random element.  This element is produced by
having the directory authorities agree on a random number once a
day~\cite{prop250}.  The random number is embedded in the consensus document and
used by clients to fetch an onion service's descriptor.  Attackers will no
longer be able to attack onion services by positioning HSDirs in the hash ring;
while they have several hours to compute a key pair that positions their HSDirs
next to the onion service's descriptor, it takes at least 96 hours to get the
HSDir flag from the directory authorities~\cite[\S~3.4.2]{dir-spec}.  We expect
this design change to disincentivize attackers from manipulating their keys to
attack onion services.

\subsection{Disclosure of results}
Throughout our research effort we were in close contact with Tor developers and
shared preliminary results.  Once we wrote up our findings in a technical
report we brought it to the broader Tor community's attention by sending an
email to the tor-dev mailing list~\cite{Roberts2017a}.  On top of that we
adopted open science practices and wrote both our code and paper in the open,
allowing interested parties to immediately learn about any progress.
