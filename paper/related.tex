\section{Related work}
In 2012, Lenstra \ea~\cite{Lenstra2012a} and Heninger \ea~\cite{Heninger2012a}
independently analyzed a large set of RSA public keys and discovered that many
keys shared prime factors, allowing an attacker to efficiently compute the
corresponding private keys.  The researchers showed that the root cause was weak
randomness at the time of key generation.  Many Internet-connected devices lack
entropy sources, resulting in predictable keys.

One year later, Bernstein \ea~\cite{Bernstein2013a} showed similar flaws in
Taiwan's national ``Citizen Digital Certificate'' database.  Among more than two
million 1024-bit RSA keys, the authors discovered 184 vulnerable keys, 103 of
which shared prime factors.  The authors could break the remaining 81 keys by
applying a Coppersmith-type partial-key-recovery attack.

Valenta \ea~\cite{Valenta2016a} optimized popular implementations for integer
factorization, allowing them to factor 512-bit RSA public keys on Amazon EC2
under four hours for only \$75.  The authors then moved on to survey the RSA key
sizes that are used in popular protocols such as HTTPS, DNSSEC, and SSH,
discovering numerous 512-bit keys of only 512 bits.

Most recently, in 2016, Hastings \ea~\cite{Hastings2016a} revisited the problem
of weak keys and investigated how many such keys remain on the Internet four
years after the initial studies.  The authors found that many vendors and device
owners never patched vulnerable devices.  To make matters worse, the number of
vulnerable devices has \emph{increased} since 2012.

In this work, we apply Lenstra \ea's and Heninger \ea's work to archived RSA
keys of Tor relays.  Drawing on a dataset that comprises almost four million
keys, we used Heninger and Halderman's \texttt{fastgcd} tool~\cite{fastgcd} to
search for shared prime factors, and manually inspected the vulnerable keys that
our work exposed.
