\section{Method}
\label{sec:method}
In this section, we discuss how we drew on a publicly-available dataset
(Section~\ref{sec:data-collection}) and used Heninger and Halderman's 
fastgcd~\cite{fastgcd} tool to analyze the public keys that 
we extracted from this dataset (Section~\ref{sec:vulnerable-keys}).

\subsection{Data collection}
\label{sec:data-collection}
The Tor Project archives data about Tor relays on its CollecTor
platform~\cite{collector}, allowing researchers to learn what relays were online
at any point in the past.  Drawing on this data source, we compiled a set of RSA
keys by downloading all server descriptors from December 2005 to December 2016 
and extracting the identity and onion keys with the Stem Python
library~\cite{stem}.  Table~\ref{tab:dataset} provides an overview of the
resulting dataset---approximately 200 GB of unzipped data.  Our 3.7 million
public keys span eleven years and were created on one million IP addresses.

\begin{table}[t]
	\centering
	\begin{tabular}{l r}
	\toprule

	First key published & 2005-12 \\
	Last key published & 2016-12 \\

	\midrule

	Number of relays (by IP address) & 1,083,805 \\
	Number of onion keys & 3,174,859 \\
	Number of identity keys & 588,945 \\
	Total number of public keys & 3,763,804 \\

	\bottomrule
	\end{tabular}
	\caption{An overview of our RSA public key dataset.}
	\label{tab:dataset}
\end{table}

\subsection{Finding vulnerable keys}
\label{sec:vulnerable-keys}
To detect weak, potentially factorable keys in the Tor network, we used Heninger
and Halderman's tool, fastgcd~\cite{fastgcd}, which takes as input a set
of moduli from public keys and then computes the pair-wise greatest common
divisor of these moduli.  Fastgcd's C implementation is based on a
quasilinear-time algorithm for factoring a set of integers into their co-primes.
We used the PyCrypto library~\cite{pycrypto} to turn Tor's PKCS\#1-padded,
PEM-encoded keys into fastgcd's expected format, which is hex-encoded
moduli.  Running fastgcd over our dataset took less than 20 minutes on
a machine with dual, eight-core 2.8 GHz Intel Xeon E5 2680 v2 processors with
256 GB of RAM.

Fastgcd benefits from having a moduli pool as large as possible
because it allows the algorithm to draw on a larger factor base to use on each
key~\cite{Heninger2012a}.  To that end, we reached out to Heninger's group at
the University of Pennsylvania, and they graciously augmented their 129 million
key dataset with our 3.6 million keys and subsequently searched for shared
factors.  The number of Tor weak keys did not go up, but this experiment gave us
more confidence that we had not missed weak keys.
