\section{Results}

\subsection{Unusual exponents}
\label{sec:unusual-exponents}
Among all our 3.7 million keys, 122 have a public exponent other than 65,537,
which is hard-coded in the Tor source code.  Two exponents---35 and 18,177---are
smaller than 65,537, and all others are larger, with the largest exponent being
35,994,097,691.  Ten of these relays have a shared modulus, which we discuss in
Section~\ref{sec:shared-moduli}.  In Appendix~\ref{sec:full-unusual-exponents},
we show all relays with unusual exponents in detail.


\subsection{Shared prime factors}

\subsection{Shared moduli}
\label{sec:shared-moduli}
In addition to finding shared prime factors, we discovered relays that shared a
\emph{modulus}, giving them the ability to calculate each other's private keys.
Table~\ref{tab:moduli} illustrates these ten relays, clustered into four groups
that each share a modulus.  The table further shows the relays' truncated,
four-byte fingerprint, IP addresses, and RSA exponents.  Note that the Tor
client hard-codes the RSA exponent to 65,537~\cite[\S~0.3]{torspec}, a
recommended value that is resistant to attacks against low public
exponents~\cite[\S~4]{Boneh1999a}.  Any other value indicates non-standard key
generation.  All IP addresses were hosted by OVH, a popular French hosting
provider, and some of the IP addresses hosted two relays, as our color coding
indicates.  Finally, each group shared a four- or five-digit prefix in their
fingerprints.  We believe that a single attacker controlled all these relays
with the intention to manipulate the distributed hash table that powers onion
services~\cite{Biryukov2013a}---the shared fingerprint prefix is an indication.
We suspect that the attackers iterated over exponents to come up with the shared
prefix.  The Tor Project informed us that it discovered and blocked these relays
in August 2014~\cite{tor-priv1} when they first came online.

\begin{table}[t]
	\centering
	\begin{tabular}{l l r}
	\toprule
	Short fingerprint & IP address & Exponent \\
	\midrule

	\texttt{\setlength{\fboxsep}{0pt}%
	\colorbox{highlight1}{\strut 838A}296A} & 188.165.164.163 &
	1,854,629 \\

	\texttt{\setlength{\fboxsep}{0pt}%
	\colorbox{highlight1}{\strut 838A}305F} &
	{\setlength{\fboxsep}{0pt}\colorbox{highlight3}{\strut 188.165.26.13}} &
	718,645 \\

	\texttt{\setlength{\fboxsep}{0pt}%
	\colorbox{highlight1}{\strut 838A}71E2} &
	{\setlength{\fboxsep}{0pt}\colorbox{highlight2}{\strut 178.32.143.175}} &
	220,955 \\

	\midrule

	\texttt{\setlength{\fboxsep}{0pt}%
	\colorbox{highlight1}{\strut 2249E}B42} &
	{\setlength{\fboxsep}{0pt}\colorbox{highlight3}{\strut 188.165.26.13}} &
	4,510,659 \\

	\texttt{\setlength{\fboxsep}{0pt}%
	\colorbox{highlight1}{\strut 2249E}C78} &
	{\setlength{\fboxsep}{0pt}\colorbox{highlight2}{\strut 178.32.143.175}} &
	1,074,365 \\

	\midrule

	\texttt{\setlength{\fboxsep}{0pt}%
	\colorbox{highlight1}{\strut E1EF}A388} & 188.165.3.63 &
	18,177 \\

	\texttt{\setlength{\fboxsep}{0pt}%
	\colorbox{highlight1}{\strut E1EF}8985} &
	{\setlength{\fboxsep}{0pt}\colorbox{highlight4}{\strut 188.165.138.181}} &
	546,019 \\

	\texttt{\setlength{\fboxsep}{0pt}%
	\colorbox{highlight1}{\strut E1EF}9EB8} &
	{\setlength{\fboxsep}{0pt}\colorbox{highlight5}{\strut 5.39.122.66}} &
	73,389 \\

	\midrule

	\texttt{\setlength{\fboxsep}{0pt}%
	\colorbox{highlight1}{\strut 410B}A17E} &
	{\setlength{\fboxsep}{0pt}\colorbox{highlight4}{\strut 188.165.138.181}} &
	1,979,465 \\

	\texttt{\setlength{\fboxsep}{0pt}%
	\colorbox{highlight1}{\strut 410B}B962} &
	{\setlength{\fboxsep}{0pt}\colorbox{highlight5}{\strut 5.39.122.66}} &
	341,785 \\

	\bottomrule
	\end{tabular}

	\caption{Four groups of relays that have a shared modulus.  All relays
	further share a fingerprint prefix in groups of two or three, presumably to
	manipulate Tor's distributed hash table.}
	\label{tab:moduli}
\end{table}

\subsection{Identifying the victims}
We have now presented several relay groups that positioned themselves in Tor's
hash ring to control an onion service.  But what onion services were targeted?

\begin{equation}
\begin{split}
\textit{secret-id-part} = \textit{SHA-1}(& \textit{time-period} \mid \\
                                         & \textit{descriptor-cookie} \mid \\
                                         & \textit{replica}) \\
\textit{descriptor-id} =  \textit{SHA-1}(& \textit{permanent-id} \mid \\
                                         & \textit{secret-id-part})
\end{split}
\end{equation}
