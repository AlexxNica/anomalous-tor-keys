\section{Introduction}
Tor is an anonymity network.  

In order to achieve its goals, Tor utilizes various cryptosystems, 
one of which is the RSA cryptosystem.  Previous research has 
shown that weak RSA public keys exist in the wild, which puts users at 
risk for decryption attacks.  

In a perfect world, Tor relay operators should run the Tor software and have it 
generate RSA keys properly. However, we found that there were thousands of keys 
that either did not conform to the Tor Specification or were weak.  More 
specifically, we found RSA public keys that share prime factors, 
that share the same modulus, and that have unusual exponents.

After discovering that there were indeed weak RSA public keys in Tor, we wanted 
to know whose relays were weak, when they were weak, and how many people might 
have used those weak relays.

Digging deeper, we discovered that many of the relays with weak keys had been owned 
by Tor researchers who were conducting experiments on . . . There are other 
relays, though, that we have no clue about.

We make the following contributions:
\begin{itemize}
  \item We find weak keys in Tor (including keys with shared moduli 
    and unusual exponents, which was previously unexplored--is that true?
    Did we find repeated keys?)
  \item We investigate the source(?) of the weak keys through manual inspection and by 
    contacting the relays' owners
  \item We demonstrate that we were indeed able to calculate the private keys of 
    some of the relays.
  \item We call attention to how researchers' experiments can put Tor users'
    information at risk
\end{itemize}

The rest of this paper details our project.  
In Section 2 we provide background information.  In Section 3 we discuss 
related work.  In Section 4 we describe our methods.  In Section 5 we 
present our results.  In Section 6 we discuss our results.  And in Section 7 
we conclude.
