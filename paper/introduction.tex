\section{Introduction}
Tor is an anonymity network.  It utilizes various cryptosystems, 
one of which is the RSA cryptosystem.  Previous research has 
shown that weak RSA public keys exist in the wild, which puts users at 
risk for decryption attacks.  

In a perfect world, Tor relay operators should run the Tor software and have it 
generate RSA keys properly. However, we found that there were thousands of keys 
that either did not conform to the Tor Specification or were weak.  That is, we 
found RSA public keys that share prime factors, we found RSA public keys that 
share the same modulus, and we found RSA public keys that have unusual exponents.

Say something about the fastgcd, 129M moduli process here?

Digging deeper, we discovered that many of the relays with weak keys had been owned 
by Tor researchers who were conducting experiments on . . . There are other 
relays, though, 

Our contributions include:
\begin{itemize}
  \item We find weak keys in Tor (including keys with shared moduli 
    and unusual exponents, which was previously unexplored--is that true?
    Did we find repeated keys?)
  \item We investigate the source(?) of the weak keys through manual inspection and by 
    contacting the relays' owners
  \item We call attention to how researchers' experiments can put Tor users
    information at risk
\end{itemize}
