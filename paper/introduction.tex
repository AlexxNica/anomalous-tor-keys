\section{Introduction}
Having seen hundreds of thousands of relays come and go over the last decade,
the Tor network is one of the largest volunteer-run anonymity networks.  To
implement onion routing, all the relays maintain several RSA key pairs, 
the most important one being a medium-term key that rotates occasionally 
and a long-term key that ideally never changes.  
Most relays run The Tor Project's reference C
implementation on dedicated Linux systems, but some run third-party
implementations or operate constrained systems such as Raspberry Pis which
raises the question of whether these machines managed to generate safe keys
upon bootstrapping.  Past work has investigated the safety of keys in TLS and
SSH servers~\cite{Heninger2012a} and  nation-wide
databases~\cite{Bernstein2013a}, as well as POP3S, IMAPS, and SMTPS
servers~\cite{Hastings2016a}.  In this work, we study the Tor network and pay
particular attention to Tor-specific aspects such as onion services.

Relays with weak cryptographic keys can pose a significant threat to Tor users.
The exact impact depends on the type of key that is vulnerable.  In the best
case, an attacker only manages to compromise the TLS layer that protects Tor
cells, which are also encrypted.  In the worst case, an attacker compromises a
relay's long-term ``identity key,'' allowing her to impersonate the relay.
To protect Tor users, we need methods to find relays with vulnerable keys and
remove them from the network before adversaries can exploit them.

Drawing on a publicly-archived dataset of 3.7 million RSA public
keys~\cite{collector}, we set out to analyze these keys for weaknesses and
anomalies: we looked for shared prime factors, shared moduli, 
and non-standard RSA exponents.
To our surprise, we found more than 3,000 keys with shared prime factors, most
belonging to a 2013 research project~\cite{Biryukov2013a}.  Ten relays in
our dataset shared a modulus, suggesting manual interference with the key
generation process.  Finally, we discovered 122 relays whose RSA exponent
differed from Tor's hard-coded exponent.  Most of these relays were meant to
manipulate Tor's distributed hash table (DHT), in an attempt to attack onion
services as we discuss in Section~\ref{sec:targeted-onion-services}.  To learn
more, we implemented a tool---itos\footnote{The name is an acronym for
``identifying targeted onion services.''}---that simulates how onion services
are placed on the DHT, revealing four onion services that were targeted by some
of these malicious relays.  Onion service operators can use our tool to monitor
their services' security, \eg, a newspaper can make sure that its SecureDrop
deployment---which uses onion services---is safe~\cite{securedrop}.

The entities responsible for the incidents we uncovered are as diverse as the
incidents themselves: researchers, developers, and actual adversaries were all
involved in generating key anomalies.  By looking for information that relays
had in common, such as similar nicknames, IP address blocks, uptimes, and port
numbers, we were able to group the relays we discovered into clusters that were
likely operated by the same entities, shedding light on the anatomy of
real-world attacks against Tor.

We publish all our source code and data, allowing third parties such as The Tor
Project to continuously check the keys of new relays and alert developers if
any of these keys are vulnerable or non-standard.\footnote{Our project page is
available online but we redacted the URL for anonymization.}  Tor
developers can then take early action and remove these relays from the network
before adversaries get the chance to take advantage of them.  In summary, we
make the following three contributions:
\begin{itemize}
	\item We analyze a dataset consisting of 3.7 million RSA public keys for
		weak and non-standard keys, revealing thousands of affected keys.

	\item We characterize the relays we discovered, show that many were
		likely operated by a single entity, and uncover four onion services that
		were likely targeted.

	\item Given a set of malicious Tor relays that served as ``hidden service
		directories,'' we develop and implement a method that can reveal what
		onion services these relays were targeting.
\end{itemize}

The rest of this paper details our project.  In Section~\ref{sec:background}, we
provide background information, followed by Section~\ref{sec:related} where we 
discuss related work.  In Section~\ref{sec:method}, we describe our method,
and Section~\ref{sec:results} presents our results.  We discuss our work in
Section~\ref{sec:discussion} and conclude in Section~\ref{sec:conclusion}.
