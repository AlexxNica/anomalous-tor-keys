\section{Background}
We now provide brief background on both the RSA cryptosystem and how the Tor
network employs RSA.

\subsection{The RSA cryptosystem}
The RSA public key cryptosystems is comprised of two keys: a public encryption
key and a privately held decryption key. The encryption key---or RSA public
key---is comprised of a pair of positive integers, an exponent $e$ and a modulus
$N$. The modulus $N$ is the product of two large, random prime numbers $p$ and
$q$. The corresponding decryption key---or RSA private key---is comprised of the
positive integer pair $d$ and $N$ where $d = e^{-1}$ mod $(p - 1)(q - 1)$.  The
decryption exponent $d$ is efficient to compute if the factorization of $N$ is
known.

The security of RSA rests upon the difficulty of factorizing $N$ into its prime
factors $p$ and $q$.  While factoring $N$ is impractical given sufficiently
large prime factors, the greatest common divisor (GCD) of \emph{two moduli} can
be computed in mere microseconds.  Consider two distinct RSA moduli $N_1 = pq_1$
and $N_2 = pq_2$ that share the prime factor $p$.  An attacker could quickly and
easily compute the GCD of $N_1$ and $N_2$ in order to obtain $p$, then divide
the moduli by $p$ to determine $q_1$ and $q_2$, thus compromising the private
key of both key pairs.  Therefore, it is crucial that both $p$ and $q$ are
determined using a strong random number generator.

\subsection{The Tor network}
As of January 2017, the Tor network consists of almost 7,000
relays~\cite{tormetrics}.  Each of these relays maintains RSA, Curve25519, and
Ed25519 key pairs~\cite[\S~1.1]{torspec}.  We are only interested in the
following three RSA key pairs that relays have.  All three key pairs use
1024-bit keys.

\begin{description}
    \item[Identity key]  Relays have a long-term identity key that they only use
        to sign documents and certificates.  Relays are frequently referred to
        by their fingerprint, which is a hash over their identity key.  The
        compromise of an identity would allow an attacker to entirely
        impersonate a relay, \eg, by publishing spoofed descriptors.
    \item[Onion key]  Relays use medium-term onion keys to decrypt cells when
        circuits are created.  The RSA-based onion key is only used in the Tor
        Authentication Protocol that is now superseded by the ntor handshake.  A
        compromised onion key allows the attacker to read the content of cells
        until the key pair rotates.
    \item[Connection key] The short-term connection keys protect the connection
        between relays using TLS.  Connection keys are rotated at least once a
        day.  The TLS connection provides defense in depth.  If compromised, an
        attacker is able to read the cells that are exchanged between Tor
        relays.
\end{description}

In our work, we only consider the three 1024-bit RSA key pairs that each relay
has.  The Tor Project is archiving these key pairs since more than ten years,
allowing us to draw on a rich data set~\cite{collector}.
