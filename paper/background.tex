\section{Background}
\label{sec:background}
We now provide brief background on both the RSA cryptosystem and how the Tor
network employs RSA.

\subsection{The RSA cryptosystem}
The RSA public key cryptosystem is comprised of two keys: a public encryption
key and a privately held decryption key \cite{rivest1978}. The encryption key,
or ``RSA public key,'' is comprised of a pair of positive integers: an exponent
$e$ and a modulus $N$. The modulus $N$ is the product of two large, random prime
numbers $p$ and $q$. The corresponding decryption key, or ``RSA private key,''
is comprised of the positive integer pair $d$ and $N$ where $d = e^{-1}$ mod $(p
- 1)(q - 1)$.  The decryption exponent $d$ is efficient to compute if e and the
factorization of $N$ are known.

The security of RSA rests upon the difficulty of factorizing $N$ into its prime
factors $p$ and $q$.  While factoring $N$ is impractical given sufficiently
large prime factors, the greatest common divisor (GCD) of \emph{two moduli} can
be computed in mere microseconds.  Consider two distinct RSA moduli $N_1 = pq_1$
and $N_2 = pq_2$ that share the prime factor $p$.  An attacker could quickly and
easily compute the GCD of $N_1$ and $N_2$ in order to obtain $p$, then divide
the moduli by $p$ to determine $q_1$ and $q_2$, thus compromising the private
key of both key pairs.  Therefore, it is crucial that both $p$ and $q$ are
determined using a strong random number generator.

Even though the naive GCD algorithm is very efficient, our dataset consists of
more than 3.7 million keys and naively computing the GCD of every pair would
take more than 3 years of computation (assuming 15 $\mu$s per pair). Instead,
we use the fast pairwise GCD algorithm by Bernstein~\cite{Bernstein04} which
can perform the computation at hand in just a few minutes.

\subsection{The Tor network}
The Tor network is among the most popular tools for digital privacy and
anonymity. As of January 2017, the Tor network consists of almost 7,000
volunteer-run relays~\cite{tormetrics}.

Each of these relays maintains RSA, Curve25519, and Ed25519 key
pairs~\cite[\S~1.1]{torspec} to authenticate and protect client traffic. In this
work, we will test the strength of these RSA keys.  We leave the analysis of the
other key types for future work.  Each Tor relay has the following three RSA
1024-bit keys:

\begin{description}
    \item[Identity key] Relays have a long-term identity key that they use only
      to sign documents and certificates.  Relays are frequently referred to by
      their fingerprints, which are hashes over their identity keys.  The
      compromise of an identity key would allow an attacker to entirely
      impersonate a relay by publishing spoofed descriptors signed by the
      forged identity key.

    \item[Onion key]  Relays use medium-term onion keys to decrypt cells when
        circuits are created.  The onion key is only used in the Tor
        Authentication Protocol that is now superseded by the \texttt{ntor} 
        handshake.  A compromised onion key allows the attacker to read 
        the content of cells until the key pair rotates.

    \item[Connection key] The short-term connection keys protect the connection
        between relays using TLS.  Connection keys are rotated at least once a
        day.  The TLS connection provides defense in depth.  If compromised, an
        attacker is able to see the enciphered cells that are exchanged between Tor
        relays.
\end{description}

In our work we consider the identity keys and onion keys that each relay 
has because the Tor Project has been archiving the public part of the 
identity and onion keys for more than 10 years, allowing us to draw on a 
rich data set~\cite{collector}. The Tor project does not archive the 
connection keys since they have short-term use and are not found
in the consensus or router descriptors.

In addition to client anonymity, the Tor network allows operators to set up
anonymous servers, typically called onion services.  A subset of all Tor relays,
the so-called hidden service directories (HSDirs), together comprise a
distributed hash table that stores the information that is necessary to connect
to an onion service.  These HSDirs are a particularly attractive target for
adversaries as they learn about new onion services they are set up in the Tor
network.
